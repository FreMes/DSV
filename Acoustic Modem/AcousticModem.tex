%%This is a very basic article template.
%%There is just one section and two subsections.
\documentclass[a4paper,11pt]{article}

\usepackage{titling}
\newcommand{\subtitle}[1]{
  \posttitle{
    \par\end{center}
    \begin{center}\large#1\end{center}
    \vskip0.5em}
}

\title{ Acoustic Modem Project:}
\subtitle{Audio Playback, recording and analysis}
\author{Frederic Mes, Wim Marynissen}
 \date{\today}


\begin{document}
\maketitle

\section{Exercise 1: Audio playback/recording}
Done.

%\subsection{Another subtitle}

\section{Exercise 2: Time-frequency analysis of recorded signals}

2.	There's some noise besides the 400Hz frequency.

3.	The DFT size determines the time interval of which a spectrum is calculated.
	(+ opgeschreven uitleg)
	
4.	There are other frequencies present. These are harmonic frequencies, they are
multiples of the original frequency.

5.	In the spectrogram of the transmitted signal, there's a DC component. But in
the recorded signal this DC component has faded. 

6.	Clipping amplifies some of the harmonic frequencies. This could be harmful
for the acoustic modem for harmonic signals could interfere with the correct
signal. (?)

8.	?

9. Possible relevant information: The noise present in the environment is not
white noise.

10. It doesn't change over time. This is relevant because we want a
time-invariant system.

11. There's more background noise and a weaker signal. (?)

\section{Exercise 3: Time-frequency analysis of recorded signals}

1. ?

4. The maximum data transfer rate for error-free transmission, in bits per
second.

\end{document}
