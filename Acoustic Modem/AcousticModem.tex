%%This is a very basic article template.
%%There is just one section and two subsections.
\documentclass[a4paper,11pt]{article}

\usepackage{titling}
\newcommand{\subtitle}[1]{
  \posttitle{
    \par\end{center}
    \begin{center}\large#1\end{center}
    \vskip0.5em}
}

\title{ Acoustic Modem Project:}
\subtitle{Audio Playback, recording and analysis}
\author{Frederic Mes, Wim Marynissen}
 \date{\today}


\begin{document}
\maketitle

\section{Exercise 1: Audio playback/recording}
Done.

%\subsection{Another subtitle}

\section{Exercise 2: Time-frequency analysis of recorded signals}

\end{document}




%
% \documentclass[a4paper,11pt]{article}
% \usepackage[dutch]{babel}
% \usepackage{amsmath}
% \usepackage{graphicx}
% \usepackage{enumerate}
% \usepackage{epstopdf}
% \usepackage{subfig}
% \usepackage{algorithmic}
% \usepackage{algorithm}
% \usepackage{amssymb}
% \usepackage{epsfig}
%   
%   
% \usepackage[margin=1.4in,vmargin=1.3in]{geometry}
% \abovecaptionskip 0.1in
% \belowcaptionskip 0.25in
%   
% \title{\huge Netwerksimulatie: verslag sessie 2: \\ ``Small UDP streams''}
% \author{Frederic Mes, Wim Marynissen}
% \date{vrijdag 4 mei 2012}
%   
% \newcommand{\dx}{\Delta x}
% \newcommand{\dy}{\Delta y}
% \newcommand{\dz}{\Delta z}
% \newcommand{\dt}{\Delta t}
%   
% \usepackage{listings}
% \usepackage{color}
% \usepackage{xcolor}
% \usepackage{textcomp}
% \definecolor{listinggray}{gray}{0.8}
% \definecolor{lbcolor}{rgb}{0.9,0.9,0.9}
% \lstset{
%     backgroundcolor=\color{lbcolor},
%     tabsize=4,
%     rulecolor=,
%     language=tcl,
%         basicstyle=\small,
%         upquote=true,
%         aboveskip={1.5\baselineskip},
%         columns=fixed,
%         showstringspaces=false,
%         extendedchars=true,
%         breaklines=true,
%         prebreak = \raisebox{0ex}[0ex][0ex]{\ensuremath{\hookleftarrow}},
%         frame=single,
%         showtabs=false,
%         showspaces=false,
%         showstringspaces=false,
%         identifierstyle=\ttfamily,
%         keywordstyle=\color[rgb]{0,0,1},
%         commentstyle=\color[rgb]{0.133,0.545,0.133},
%         stringstyle=\color[rgb]{0.627,0.126,0.941},
%         morekeywords={$ns}
% }
%   
%   
% \renewcommand{\lstlistingname}{Codefragment}
%   
%   
%   
% \begin{document}
% \maketitle
%   \vspace{-21pt}
% \noindent
% \textit{Kleine UDP stromen worden vaak gebruikt bij online computer games. De
% throughput-karakteristieken voor een \emph{Tahoe TCP} connectie in parallel met
% vele van deze UDP stromen toont een aantal verschillen met die van een
% \emph{Reno TCP}. In dit verslag bespreken we de verschillen met behulp van plots
% van de Throughput, het Congestion Window en de Slow Start Threshold.}
%   
%   
% \begin{figure}[h]
% \section{Tahoe TCP Implementatie}
% \vspace{-5pt}
% Eerst en vooral dient te worden opgemerkt dat UDP geen congestion control heeft.
% Bij competitie neemt het UDP verkeer zoveel bandbreedte in als nodig. TCP heeft
% wel congestion control om packet loss te vermijden. 
% De hoeveelheid bandbreedte ingenomen door de 50 UDP streams is niet groot.
%  Toch zijn er drops in de throughput van de TCP flow, zie figuur
%  ~\ref{throughput_n1_tcp}. Dit komt door verloren pakketten en het congestion
%  avoidance algoritme van TCP: bij het verloren gaan van pakketten verlaagt de
%  throughput als gevolg van het dalen van het \textit{Congestion Window}.
% De \textit{Window Size} van de TCP is voor deze simulatie op 80 gezet. Hierdoor
% gaat de TCP connectie een grote bandbreedte innemen. De TCP connectie bezet bijna de volledige bufferruimte 
% nog voor de UDP streams pakketten beginnen te sturen.
%         \begin{center}
%         \includegraphics[width=0.8\textwidth]{throughput_n1_tcp.png}
%         \end{center}
%         \vspace{-13pt}
%         \caption{De throughput van de verschillende flows bij Tahoe TCP.}
%         \label{throughput_n1_tcp}
% \end{figure}
% \clearpage
%   
% \noindent
% Het TCP protocol voorziet minstens twee vensters voor de zender. Een daarvan is
% het \textit{Congestion Window}. Dit is het aantal bytes dat op een ogenblik mag
% aanwezig zijn (aan het verzenden zijn) doorheen een connectie. Bij packet loss wordt het
% congestion window verlaagt. \\ \\
% Er zijn twee fases: \textit{Slow Start} en \textit{Congestion Avoidance}. Bij
% slow start (hoewel de naam anders doet vermoeden) zal het congestion venster
% exponentieel stijgen. Bij congestion avoidance gebeurt dit lineair. De
% \textit{Slow Start Threshold} is de grens tussen deze twee fasen.
% Bij een packet loss wordt deze grenswaarde gelijk gesteld aan
% de gehalveerde waarde van het congestion window, en het congestion window terug op 1 gezet. Zie figuur ~\ref{congw_tcp}.
% Bij het ontvangen van drie opeenvolgende ACK's van eenzelfde pakket wordt het
% pakket verloren verondersteld. Dan gebeurt een \textit{Fast Retransmission},
% waardoor de tijd dat de zender moet wachten voor het retransmitten bij een verloren segment gereduceerd wordt.
%   
% %
% %Op figuur~\ref{congw_tcp} zien we dat het \textit{Congestion Window} telkens op
% %1 wordt gezet nadat er drops zijn gedetecteerd. Daarna zal dit venster opnieuw
% % stijgen.
% %Dit gebeurt volgens het 'Fast Retransmit' mechanisme, nl. het congestion window
% % zal exponentieel stijgen tot de slow start threshold bereikt is, daarna zal hij nog lineair toenemen tot de volgende drop. \\
% %De slow start threshold is dus een grens in het slow start algoritme. Deze
% % grens wordt bij een drop gelijk gesteld aan de helft van de \textit{Congestion Window}. Bij de eerste drop werd de threshold gelijk gesteld aan de
% %helft van de vorige waarde. \\
% \begin{figure}[h]
%         \begin{center}
%         \includegraphics[width=0.8\textwidth]{congw_ssth_tcp.png}
%         \end{center}
%         \caption{Congestion window en slow start threshold bij Tahoe TCP.}
%         \label{congw_tcp}
% \end{figure}
%   
% \clearpage
% \section{Reno TCP Implementatie}
% Bij Reno moet onderscheid gemaakt worden tussen verschillende wijzen van packet
% loss. 
% Bij het ontvangen van drie dezelfde ACK's (bevestigingen) van eenzelfde
% pakket wordt het pakket verloren verondersteld. Dan zal het congestion
% window gelijk gesteld worden aan de slow start threshold of wordt het dus
% gehalveerd. Er gebeurt een fast retransmission net als bij Tahoe TCP. Vervolgens
% gebeurt \textit{Fast Recovery}. Hierin zal TCP het missende pakketje
% heruitzenden en wachten op een bevestiging van het gehele transmit venster voordat terug wordt overgegaan naar congestion avoidance.\\
% Wanneer een ACK een time out heeft, zal (net zoals bij Tahoe) overgegaan worden
% op \textit{Slow Start}. Op figuur ~\ref{congw_reno} bij ongeveer 17 seconden en bij 23 seconden zien we dit gebeuren. Het congestion window wordt dan op 1
% gezet en de slow start threshold wordt (in tegenstelling tot de werking bij
% Tahoe) gehalveerd.
% \\
% Deze vorm van TCP vermijdt dus \textit{Slow Start}, behalve bij de opstart en
% bij een timeout van een bevestiging.
% \\
% \begin{figure}[h]
%         \begin{center}
%         \includegraphics[width=0.8\textwidth]{cong_ssth_reno.png}
%         \end{center}
%         \caption{Congestion window en slow start threshold bij Reno TCP.}
%         \label{congw_reno}
% \end{figure}
%   
%   
%   
% \clearpage
% \noindent
% Op figuur ~\ref{throughput_n1_reno} zien we dat de throughput drops nu
% veel minder diep zijn dan bij de Tahoe TCP implementatie.
% Dit heeft te maken met het besproken \textit{Fast Recovery} mechanisme van Reno
% TCP, waarbij de pakketstroom niet wordt geleegd (zoals bij Tahoe TCP) maar wel
% gereduceerd. (zie boven) \\
% \begin{figure}[h]
%         \begin{center}
%         \includegraphics[width=0.8\textwidth]{throughput_n1_reno.png}
%         \end{center}
%         \caption{De throughput van de verschillende flows bij Reno TCP.}
%         \label{throughput_n1_reno}
%          
%         \noindent
% Tweemaal zakt de throughput volledig tot 0. Dit gebeurt bij het te
% laat ontvangen van een acknowledgement, een 'timeout'. Dan daalt het congestion
% window tot op 1 zoals eerder besproken.
% \end{figure}
%   
%   
%   
%   
% \clearpage
% \section{Appendix: TCL-code}
% \begin{lstlisting}
% #Create a simulator object:
% set ns [new Simulator]
%   
% #Kleuren van flows:
% $ns color 1 Blue
% $ns color 2 Red
%   
% #Nodige bestanden openen:
% set tracefile1 [open /tmp/wimfred/out_oef2b.tr w]
%   
% set filebursts_tcp [open filebursts_tcp w]
% set filecwnd_tcp [open filecwnd_tcp w]
% set filessth_tcp [open filessth_tcp w]
% $ns trace-all $tracefile1
%   
% set nf [open /tmp/wimfred/out_oef2b.nam w]
% $ns namtrace-all $nf
%   
% #Define a 'finish' procedure:
% proc finish {} {
%         global ns nf tracefile1
%         $ns flush-trace
%     close $tracefile1
%         #Close the NAM trace file
%         close $nf
%         #Execute NAM on the trace file
%         exec nam /tmp/wimfred/out_oef2b.nam &
%         exit 0
% }
%   
% #Create four nodes:
% set n0 [$ns node]
% set n1 [$ns node]
% set n2 [$ns node]
% set n3 [$ns node]
% set n4 [$ns node]
% set n5 [$ns node]
%   
% #Create links between the nodes:
% $ns duplex-link $n2 $n0 10Mb 10ms DropTail
% $ns duplex-link $n4 $n0 10Mb 10ms DropTail
% $ns duplex-link $n0 $n1 10Mb 10ms DropTail
% $ns duplex-link $n1 $n3 10Mb 10ms DropTail
% $ns duplex-link $n1 $n5 10Mb 10ms DropTail
%   
% #Queue groottes instellen:
% $ns queue-limit $n0 $n1 20
% $ns queue-limit $n4 $n0 2000
%   
% #Give node position (for NAM):
% $ns duplex-link-op $n2 $n0 orient right-down
% $ns duplex-link-op $n4 $n0 orient right-up
% $ns duplex-link-op $n0 $n1 orient right
% $ns duplex-link-op $n1 $n3 orient right-up
% $ns duplex-link-op $n1 $n5 orient right-down
%   
% #Random generator aanmaken:
% set rng1 [new RNG]
% $rng1 seed 0
%   
% #Exponentieel verdeelde delay:
% set Delay [new RandomVariable/Exponential]
% $Delay set avg_ 0.03
% $Delay use-rng $rng1
%   
% #50 UDP connections aanmaken:
%     for {set j 0} {$j<=49} {incr j} {
% #UDP connectie:
%         set udp($j) [new Agent/UDP]
%         $ns attach-agent $n4 $udp($j)
%         set null($j) [new Agent/Null]
%         $ns attach-agent $n5 $null($j)
%         $ns connect $udp($j) $null($j)
%         $udp($j) set fid_ 2
%          
% #CBR connectie:
%         set cbr($j) [new Application/Traffic/CBR]
%         $cbr($j) attach-agent $udp($j)
%         $cbr($j) set type_ CBR
%         $cbr($j) set packetSize_ 400b
%         $cbr($j) set interval_ 100ms
%         $cbr($j) set random_ false
%     }
%   
%   
% #Start van CBR'en schedulen:
% set t 5.0
% for {set j 0} {$j<=49} {incr j} {
%     set t [expr $t + [$Delay value]]
%     $ns at $t "$cbr($j) start"
%     puts $filebursts_tcp "$j $t"
% }
%   
% #TCP connectie:
% set tcp1 [new Agent/TCP]
% $ns attach-agent $n2 $tcp1
% set sink120 [new Agent/TCPSink]
% $ns attach-agent $n3 $sink120
% $ns connect $tcp1 $sink120
% $tcp1 set fid_ 1
% $tcp1 set window_ 80
%   
% #FTP over TCP:
% set ftp1 [new Application/FTP]
% $ftp1 attach-agent $tcp1
%   
% #plotWindow procedure:
% proc plotWindow {tcpSource} {
%     global ns tcp1 filecwnd_tcp filessth_tcp
%   
%     set time 0.1
%     set now [$ns now]
%     set cwnd [$tcpSource set cwnd_]
%     set ssth [$tcpSource set ssthresh_]
%     puts $filecwnd_tcp "$now $cwnd"
%     puts $filessth_tcp "$now $ssth"
% $ns at [expr $now+$time] "plotWindow $tcpSource"}
%   
% #Procedure en ftp starten/stoppen:
% $ns at 0.1 "plotWindow $tcp1"
%   
% $ns at 0.1 "$ftp1 start"
%   
% $ns at 24.7 "$ftp1 stop"
%   
% $ns at 25.0 "finish"
%   
% #Run the simulation
% $ns run
% \end{lstlisting}
% \end{document}
